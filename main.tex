\documentclass{article}
\usepackage[utf8]{inputenc}
\usepackage[spanish]{babel}
\usepackage{listings}
\usepackage{graphicx}
\graphicspath{ {images/} }
\usepackage{cite}

\begin{document}

\begin{titlepage}
    \begin{center}
        \vspace*{1cm}
            
        \Huge
        \textbf{Ideación Proyecto Final}

        \vspace{0.5cm}
        \LARGE
        I'm Alive
            
        \vspace{1.5cm}
            
        \textbf{Sergio Alberto Giraldo Salazar y Ricardo Echeverri Cano}
            
        \vfill
            
        \vspace{0.8cm}
            
        \Large
        Despartamento de Ingeniería Electrónica y Telecomunicaciones\\
        Universidad de Antioquia\\
        Medellín\\
        Marzo de 2021
            
    \end{center}
\end{titlepage}

\tableofcontents
\newpage
\section{Idea del Juego}\label{idea}
El juego va a tratar de un androide el cual desarrollo conciencia propia y desea vivir una vida como un ser autónomo e independiente, para lograr su sueño debe escapar del laboratorio donde fue creado, ya que fue creado para ser un arma de destrucción.
\cite{Alaluzdeunabombilla}

\section{Personajes Principales} \label{personajes}
\begin{itemize}
    \item Androide (personaje principal).
    \item Segundo Androide (sólo si es multijugador).
\end{itemize}

\section{Enemigos}\label{enemigos}
\begin{itemize}
    \item Soldados: Estos enemigos atacaran al personaje bajando su barra de vida.
    \item Científicos: Estos se deben de evitar ya que si ven al personaje llamaran a un grupo de soldados.
\end{itemize}

\section{Mapa}\label{mapa}
El juego va a contar con un mapa de un laboratorio subterráneo, el cual estará dividido en 5 niveles que serían los 5 pisos del laboratorio, y para poder avanzar de nivel deberá cumplir con las respectivas misiones asignadas en cada uno o matar a los enemigos que sean necesarios.

\section{Modo de Juego}\label{modo}
\begin{itemize}
    \item \textbf{Perspectiva:} Cuarta persona.
    \item \textbf{Modo de combate:} Disparar proyectiles los cuales se detienen hasta que impacten un soldado o un muro, esquivar los proyectiles de los soldados y evitar el campo de visión de los científicos, conseguir las tarjetas para poder abrir las puertas.
\end{itemize}

\section{Nivel de Dificultad}\label{dificultad}
La dificultad ira aumentando junto con los niveles. Esto se hará incrementando la cantidad de enemigos, cadencia de fuego y punto de vida de los enemigos.

\section{Gráficos}\label{gráficos}
Se definirán al avanzar el curso.

\bibliographystyle{IEEEtran}
\bibliography{references}

\end{document}
